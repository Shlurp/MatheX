\input mathex
\setscalefactor{10}

\catcode`\@=11

\let\@xp=\expandafter
\font\sc = cmcsc9

\def\fakebold#1{\pdf@literal{2 Tr .3 w}#1\pdf@literal{0 Tr 1 w}}
\def\begincenter{\bgroup\leftskip=0pt plus 1fill \rightskip=\leftskip}
\def\endcenter{\par\egroup}

\def\flip#1{{\setbox0=\hbox{#1}\kern\wd0\pdf@literal{q -1 0 0 1 0 0 cm}\rlap{#1}\pdf@literal{Q}}}

\font\sevenrm = cmr7
\def\MatheX{Math\kern-.1ex\lower .6ex\hbox {E}\kern -.125emX}
\def\xetex{X\kern-.125em\lower.5ex\hbox{\flip{E}}\kern-.1667em\TeX}
\def\LaTeX{L\kern-.36em{\setbox0=\hbox{T}\vbox to\ht0{\hbox{\sevenrm A}\vss}}\kern-.15em\TeX}

\def\fontuse{\afterassignment\@@ \font\@@}
\def\scalefont{\fontuse \curr@font scaled }

\def\title#1{\leavevmode\vskip.1\vsize\centerline{\fontuse cmcsc10 scaled 5000 \fakebold{#1}}\bigskip}

\newcount\c@section
\newcount\c@subsection
\def\section#1{\bigskip\advance\c@section by 1 \c@subsection=0 \bgroup\fontuse cmbx10 scaled 2000 \the\c@section. #1\par\bigskip\egroup}
\def\subsection#1{\medskip\advance\c@subsection by 1 \bgroup\fontuse cmbx10 scaled 1500 \the\c@section.\the\c@subsection. #1\par\medskip\egroup}

\chardef\fslash=`\/
{\catcode`\^^M=\active%
\gdef\begincode{%
    \bgroup\everypar={}%
    \medskip%
    \setbox0=\vbox\bgroup%
    \global\dimen69=0pt%
    \tt%
    \catcode`\^^M=\active%
    \def^^M{\egroup\ifnum\wd1>\dimen69 \global\dimen69=\wd1\fi\copy1\setbox1=\hbox\bgroup}%
    \catcode`\ =\active%
    \catcode`\{=12%
    \catcode`\}=12%
    \catcode`\/=0%
    \catcode`\$=12%
    \catcode`\\=12%
    \catcode`\%=12 \setbox1=\hbox\bgroup}}
\def\endcode{%
    \egroup\egroup%
    \hfil\hbox to \dimen69{\box0}\hfil\par%
    \medskip%
    %\let\@par=\par%
    \egroup%
}

\def\icode{\bgroup\tt%
    \catcode`\ =\active%
    \catcode`\/=0%
    \catcode`\{=12%
    \catcode`\}=12%
    \catcode`\$=12%
    \catcode`\\=12%
}
\def\eicode{\egroup}

{\catcode`\<=\active 
\gdef\macrousage{\bgroup\setbox0=\hbox\bgroup\tt%
    \catcode`\ =\active%
    \catcode`\/=0%
    \catcode`\{=12%
    \catcode`\}=12%
    \catcode`\$=12%
    \catcode`\\=12%
    \catcode`\<=\active%
    \def<##1>{$\langle${\it ##1}$\rangle$}%
}}
\def\emacrousage{\egroup\par\smallskip\centerline{\box0}\egroup\par\smallskip}
\def\emacrousageB{\egroup\box0\egroup}

\def\macroname#1{{\tt\string#1}}

\chardef\lbrace="7B
\chardef\rbrace="7D

\def\macroexp#1{\par\smallskip\bgroup\leftskip=1cm\leavevmode\kern-1cm\macrousage \scantokens{#1}\emacrousageB:}
\def\emacroexp{\par\egroup\smallskip}

\def\showcase#1#2{{\openup3\jot\halign{\tt\hskip 1cm\string##\hfil&\quad\hfil##\cr
    \omit\tt\string#1: & $\displaystyle#1$\cr\noalign{\kern3pt\hrule\kern3pt}
    \displaystyle & $\displaystyle #2$\cr
    \textstyle & $\textstyle #2$\cr
    \scriptstyle & $\scriptstyle #2$\cr
    \scriptscriptstyle & $\scriptscriptstyle #2$\cr}
}}

\def\showcasevecc#1#2#3{\hbox to\hsize{\hss\vbox{\tabskip=5pt\openup3\jot%
\halign{\tt\hskip 1cm\string##\tabskip=0pt\hfil&\quad##\hfil&\quad\hfil##\tabskip=5pt\cr
    \omit\tt\string#1: & \tt\string#1 & \tt\expandafter\string\csname short\m@strip#1\endcsname%
    \cr\noalign{\kern3pt\hrule\kern3pt}
    \displaystyle      & $\displaystyle #2$      & $\displaystyle #3$\cr
    \textstyle         & $\textstyle #2$         & $\textstyle #3$\cr
    \scriptstyle       & $\scriptstyle #2$       & $\scriptstyle #3$\cr
    \scriptscriptstyle & $\scriptscriptstyle #2$ & $\scriptscriptstyle #3$\cr}
}\hss}\bigskip}

\def\showcasearrow#1#2#3#4{\hbox to\hsize{\hss\vbox{\tabskip=0pt\openup3\jot
\halign{\tt\hskip 1cm\string##\tabskip=0pt\hfil&\quad##\hfil&\quad\hfil##\hfil\quad&\quad\hfil##\tabskip=5pt\cr
    \omit\tt\string#1: & \tt\string#1 & \tt\expandafter\string\csname long\m@strip#1\endcsname &
    \tt\expandafter\string\csname x\m@strip#1\endcsname\cr\noalign{\kern3pt\hrule\kern3pt}
    \displaystyle      & $\displaystyle #2$      & $\displaystyle #3$        & $\displaystyle #4$\cr
    \textstyle         & $\textstyle #2$         & $\textstyle #3$           & $\textstyle #4$\cr
    \scriptstyle       & $\scriptstyle #2$       & $\scriptstyle #3$         & $\scriptstyle #4$\cr
    \scriptscriptstyle & $\scriptscriptstyle #2$ & $\scriptscriptstyle #3$   & $\scriptscriptstyle #4$\cr}
}\hss}\bigskip}

\def\showcaseaccent#1#2{{\openup3\jot\halign{\tt\hskip 1cm\string##\hfil&\quad\hfil##\cr
    \omit\tt\string#1: & \cr\noalign{\kern3pt\hrule\kern3pt}
    \displaystyle & $\displaystyle #2$\cr
    \textstyle & $\textstyle #2$\cr
    \scriptstyle & $\scriptstyle #2$\cr
    \scriptscriptstyle & $\scriptscriptstyle #2$\cr}
}}

\parindent=0pt
\parskip=3pt
\hoffset=\dimexpr 2cm - 1in \relax
\advance\hsize by -2\hoffset
\voffset=2\hoffset
\advance\vsize by -2\voffset

\nopagenumbers
\title{\MatheX}
\centerline{\it S. Lurp}
\centerline{\it October, 2022}

\bigskip
\hbox to \hsize{\hss\vbox{\hsize=.7\hsize
\hrule
\smallskip
\begincenter
The \MatheX{} package was created as an extension to the math macros provided by \TeX{} and \LaTeX{}.
It provides more obscure symbols not found in popular preexisting and reimpliments macros which were viewed as flawed.

A big part of the package was implemented through \macroname\pdfliteral s and other {\sc pdf} primitives.
So the \MatheX{} package is intended for use with {\sc pdf}\TeX, Lua\TeX, \xetex, and their \LaTeX{} counterparts.
Unfortunately, some macros are not supported by \xetex.
\endcenter
\smallskip
\hrule
}\hss}

\vfill\break

\footline={\hfil\folio\hfil}
\section{An Introduction to \MatheX{}}

The main motivator for creating \MatheX{} was \TeX's poor implementation of the \macroname\overrightarrow{} macro which many
times yields unsavory results.
For example \icode \overrightarrow{\hbox{ABC}}/eicode{} yields:

$$ \overrightarrow{\hbox{ABC}} $$

As you can see, the arrow overlaps with the {\tt ABC} which is undesirable.
This can be fixed by altering the \macroname\rightarrow{} macro, but I decided to make a more versatile alternative.
The \MatheX{} alternative, \macroname\vecc{}, on the other hand yields:

$$ \vecc{\hbox{ABC}} $$

Along with a few other features, \MatheX{} provides a simple interface for creating your own style of arrows.

\MatheX{} requires the current font size in order to properly scale its symbols, which must be provided after
{\tt\string\input}ing {\tt mathex.tex}.
This can be done with the \macroname\setscalefactor{} macro.
If your font is 12pt then you can load \MatheX{} like so:

\begincode
\input mathex
\setscalefactor{12}
/endcode\par

It is {\it imperative} that you set the scale factor after loading {\tt mathex} as otherwise almost none of the macros will
work.

\section{The Predefined Symbols}

This section will simply be an exhaustive list of all the predefined symbols \MatheX{} provides.

\subsection{Math Symbols}

{\tabskip=10pt plus 5pt minus 5pt\openup3\jot\halign to \hsize{{\hsize=.45\hsize#}\hfil&\hfil{\hsize=.45\hsize#}\cr
\vbox{\showcase\gprod{A\gprod B}}                       & \vbox{\showcase\dcup{A\dcup B}}\cr
\vbox{\showcase\biggprod{A\gprod\biggprod_{n=1}^N B_n}} & \vbox{\showcase\bigdcup{A\dcup\bigdcup_{n=1}^N B_n}}\cr
\noalign{\vfil\break}
\vbox{\showcase\aint{f(x) + \aint_a^b g(x)\, dx}}       & \vbox{\showcase\divs{n\divs m}}\cr
\vbox{\showcase\ndivs{n\ndivs m}}                       & \cr}}

Additionally, \macroname\lightning{} is provided as a textmode command and renders \lightning.

\subsection{Vector Symbols}

Each vector comes as a pair: the normal form and the short form.
The normal form is meant to cover longer material while the short form covers shorter material.

\bigskip
\showcasevecc\vecc{\vecc{\hbox{ABC}}}{\shortvecc{a}}
\showcasevecc\lvecc{\lvecc{\hbox{ABC}}}{\shortlvecc{a}}
\showcasevecc\overrightharp{\overrightharp{\hbox{ABC}}}{\shortoverrightharp{a}}
\vfill\break
\showcasevecc\overleftharp{\overleftharp{\hbox{ABC}}}{\shortoverleftharp{a}}
\showcasevecc\oveleftrrightvecc{\overleftrightvecc{\hbox{ABC}}}{\shortoverleftrightvecc{a}}
\showcasevecc\oveleftrrightharp{\overleftrightharp{\hbox{ABC}}}{\shortoverleftrightharp{a}}
\showcasevecc\overrightleftharp{\overrightleftharp{\hbox{ABC}}}{\shortoverrightleftharp{a}}
\showcasevecc\straightvecc{\straightvecc{\hbox{ABC}}}{\shortstraightvecc{a}}
\showcasevecc\straightlvecc{\straightlvecc{\hbox{ABC}}}{\shortstraightlvecc{a}}

\unless\ifx\pdfxform\undefined
The \macroname\constvec{} macro has the following usage:
\macrousage \constvec<vector macro>{<material>} /emacrousage
And it centers the {\tt vector macro} above {\tt material} as if it had the same height as {\tt x}, cropping anything above
that height.
So for example \icode \constvec\vecc{abc}/eicode{} yields $\constvec\vecc{abc}$.
This macro cannot be used in \xetex.
\fi
\subsection{Arrow Symbols}

Each arrow comes as a triplet: the normal form, the long form, and the extendable form.
The extendable form is similar to \macroname\xrightarrow{} and friends, an extendable arrow has the following use:
\macrousage \xarrow{<top material>}[<bottom material>] /emacrousage
And creates an extended arrow to fit both the top and bottom material.

\bigskip
\showcasearrow\varrightarrow{A\varrightarrow B}{A\longvarrightarrow B}{A\xvarrightarrow{ABC}[abc] B}
\showcasearrow\varleftarrow{A\varleftarrow B}{A\longvarleftarrow B}{A\xvarleftarrow{ABC}[abc] B}
\showcasearrow\varrightharp{A\varrightharp B}{A\longvarrightharp B}{A\xvarrightharp{ABC}[abc] B}
\showcasearrow\varleftharp{A\varleftharp B}{A\longvarleftharp B}{A\xvarleftharp{ABC}[abc] B}
\showcasearrow\varleftrightarrow{A\varleftrightarrow B}{A\longvarleftrightarrow B}{A\xvarleftrightarrow{ABC}[abc] B}
\showcasearrow\varleftrightharp{A\varleftrightharp B}{A\longvarleftrightharp B}{A\xvarleftrightharp{ABC}[abc] B}
\showcasearrow\varrightleftharp{A\varrightleftharp B}{A\longvarrightleftharp B}{A\xvarrightleftharp{ABC}[abc] B}
\showcasearrow\varmapsto{A\varmapsto B}{A\longvarmapsto B}{A\xvarmapsto{ABC}[abc] B}
\showcasearrow\varhookrightarrow{A\varhookrightarrow B}{A\longvarhookrightarrow B}{A\xvarhookrightarrow{ABC}[abc] B}
\showcasearrow\varhookleftarrow{A\varhookleftarrow B}{A\longvarhookleftarrow B}{A\xvarhookleftarrow{ABC}[abc] B}
\showcasearrow\vardoublerightarrow{A\vardoublerightarrow B}{A\longvardoublerightarrow B}{A\xvardoublerightarrow{ABC}[abc] B}
\showcasearrow\vardoubleleftarrow{A\vardoubleleftarrow B}{A\longvardoubleleftarrow B}{A\xvardoubleleftarrow{ABC}[abc] B}

\subsection{Wide Accents}
{\tabskip=20pt plus 10pt minus 5pt\halign{{\hsize=.45\hsize#}\hfil&\hfil{\hsize=.45\hsize#}\cr
\vbox{\showcaseaccent\varwidehat{\varwidehat{A}}} & \vbox{\showcaseaccent\varwidecheck{\varwidecheck{ABC}}}\cr}}
\bigskip
\hbox to\hsize{\hfil\vbox{\tabskip=20pt plus 10pt minus 5pt\showcaseaccent\varwidetilde{\varwidetilde{ABC}}}\hfil}

\unless\ifx\pdfxform\undefined
\subsection{Extendable Operators}
Extendible operators extend to the width of the material in their limits.
These operators should only be used in display mode, since they use the display modes of the operators.
They are \macroname\suum{} and \macroname\prood{}:

$$ \suum_{\hbox{abcdef}}^{\hbox{ABCDEF}} \qquad \prood_{\hbox{abcdef}}^{\hbox{ABCDEF}} $$

These are not available in \xetex.
\fi

\section{Defining Your Own Symbols}

\MatheX{} provides an interface for creating your own mathematical symbols through the use of \macroname\pdfliteral s.
This interface requires prior knowledege of drawing with PDFs.

\macroexp{\@linehead@type{<pdf code>}{<width>}} This creates a ``linehead'' which is used to cap lines, like
\macroname\@rarrow{} ($\@rarrow{.4}{2.5}$).
{\it pdf code} is the actual code used to draw the symbol, and it should be noted that all necessary transformations to the
linehead are done by \macroname\@linehead@type{} and should not be included in the code.
This includes the setting of the width and transforming the coordinate system.
The {\it width} is the width of the drawing of the {\it pdf code}.

This macro actually accepts more parameters, but they're used internally and therefore aren't necessary to explain.
Therefore the only use this macro should be for is defining line heads.
For example, the definition of \macroname\@rarrow{} is:
\begincode
\def\@rarrow {\@linehead@type{0 0 m 2 0 l 1 0 0 1 0 1.5 c 2 0 m 1 0 0 -1 0 -1.5 c S}{2}}
/endcode

The predefine lineheads are
\icode \@rarrow, \@larrow, \@rharp, \@lharp, \@rdharp, \@rlharp, \@mapcap, \@rsarrow, \@lsarrow, \@backhook, \@fronthook, \@doublerarrow, \@doublelarrow/eicode.
\emacroexp

\macroexp{\@vecc@def{<vector name>}<left cap><right cap>} This creates a vector macro, like \macroname\vecc.
This creates both the normal and short variations of the vector.
For example, the definition of the {\tt vecc} vectors is:
\begincode
\@vecc@def{vecc}\@linecap\@rarrow
/endcode
\emacroexp

\macroexp{\@arrow@def{<arrow name>}<left cap><right cap>} This creates an arrow macro, like \macroname\varrightarrow.
This creates the normal, long, and extendable versions of the arrow.
For example, the definition of the {\it varrightarrow} vectors is:
\begincode
\@arrow@def{varrightarrow}\@linecap\@rarrow
/endcode
\emacroexp

\macroexp{\@wide@accent{<pdf code>}} This creates a wide accent, like \macroname\varwidecheck.
The width of the drawing by the {\it pdf code} should be $1$, and it should be filled not stroked (since the accent is
transformed to stretch over the material beneath it).
Again this macro should only be used to define wide accents.
For example, the definition of \macroname\varwidecheck{} is:
\begincode
\def\varwidecheck{\@wide@accent{0 1.3 m .5 -.4 l 1 1.3 l 1 1.6 l .5 .3 l 0 1.6 l f}}
/endcode
\emacroexp

\macroexp{\pdf@drawing@macro{<name>}{<pdf code>}{<width>}{<height>}{<horizontal skew>}{<depth>}} This creates a text mode
symbol like \macroname\lightning.
It is important that the {\it pdf code} fits inside the box created by {\it width, height,\/} and {\it depth} since the
drawing is placed inside of an XForm and so anything outside the box will be cropped.
The {\it horizontal skew} can be used to shift the symbol so that it fits horizontally inside the box.
For example, the definition of \macroname\lightning{} is:
\begincode
\pdf@drawing@macro{lightning}       % The lightning symbol is drawn upright
    {.86603 -.5 .5 .86603 0 0 cm    % and rotated 30 degrees
    1 J 1 j .6 w
    -3 10 m -3 4.133975 l 0 5.866025 l 0 0 l -1.125 1.5 l 0 0 l 1.125 1.5 l S}
    {4.2pt}{10.5pt}{.5pt}{.9pt}
/endcode
\emacroexp

\macroexp{\pdf@drawing@math@macro{<name>}{<pdf code>}{<width>}{<height>}{<skew>}{<depth>}<style scaling>} This creates a math
mode symbol like \macroname\divs.
The first few parameters are identical in use as \macroname\pdf@drawing@macro's, and {\it style scaling} is used to set the
scaling for the symbol in different math styles.
{\it style scaling} should be three groups: the first group is the scaling used in textstyle, the second in scriptstyle, and
the third in scriptscriptstyle.
Each of these scalings should consist of two components: the fractional scaling and the decimal scaling.
So for example if we'd like to scale it by $0.6$ the scaling would be \icode {{6 /fslash 10}{0.6}}/eicode.
This is necessary since \macroname\dimexpr{} doesn't play with decimals nicely, but \macroname\pdfliteral{} requires them.
For example, the definition of \macroname\divs{} is:
\begincode
\pdf@drawing@math@macro{@divs}
    {1.3 w 1 j
    2.5 1 0 .1 re
    2.5 5 0 .1 re
    2.5 9 0 .1 re B}
    {5.4pt}{10pt}{0pt}{.2pt}
    {{1}{1}}{{7 /fslash 10}{.7}}{{11 /fslash 20}{.55}}
\def\divs{\mathrel{\@divs}}     % Make divs a relation
/endcode
\emacroexp

\macroexp{\putsym{<main symbol>}{<secondary symbol>}} This centers the {\it secondary symbol} over the {\it main symbol}, and
can be used to create symbols like \macroname\aint.
Note that doing this creates a symbol which acts like an Ord on the left side and whatever type of atom {\it main symbol} is
on the right (glue-wise).
So it may be necessary to add some math atom ``hackery'' around the \macroname\putsym{} in order to get the target glue.
For example, the definition of \macroname\aint{} is:
\begincode
\def\aint{\mathop{}\mathclose{}\putsym\int-}
/endcode
The \icode \mathop{}\mathclose{}/eicode{} makes it act like an Op on the left (the \macroname\mathclose{} removes any glue
added on the right of the \macroname\mathop).
Usually the definition is simpler, but this is slightly more complicated since \macroname\int{} has specially placed limits.
Another example, this time the definition of \macroname\bigdcup{} is:
\begincode
\def\bigdcup{\mathop{\putsym\bigcup\cdot}}
/endcode
\emacroexp

\unless\ifx\pdfxform\undefined
\macroexp{\@wide@operator{<name>}<operator>{<first cut>}{<second cut>}} This creates an extendable operator of {\it operator}
whose name is {\it name}, like \macroname\suum.
{\it first cut} is a decimal value which is where on the width of {\it operator} to make the first slice, and similar for
{\it second cut}.
The extendable part of the new operator is the area between the two cuts.
For example, the definition of \macroname\suum{} is:
\begincode
\@wide@operator{suum}\sum{.52}{.6}
/endcode
You can see where the slices are for a wide operator using the \macroname\@show@slices{} macro, for example
\begincode
\@show@slices{suum}
/endcode
gives:
$$ \@show@slices{suum} $$
These macros are not available for \xetex.
\emacroexp
\fi
\end
